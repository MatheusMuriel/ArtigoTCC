%%%%%%%%%%%%%%%%%%%%%%%%%%%%%%%%%%%%%%%%%%%%%%%%%%%%%%%%%%%%%%%%%%%%%%%%%%%%%%%%%%%%%%%%%%%%%%%

%% 2 ::: Desenvolvimento
\chapter{Desenvolvimento}
    %% Interlúdio do [Desenvolvimento] %%
        ...
    %% Fim do Interlúdio do [Desenvolvimento] %%


%% 2 ::: Desenvolvimento
%% 2.1
\section{Metodologia de Pesquisa}
    %% Interlúdio do [Metodologia de Pesquisa] %%
        Para os testes deste trabalho estão sendo usados os mesmo problemas utilizados por \cite{Kacem2002}, e que já são largamente usados na literatura. E são representados por $[j, o, m]$ em que $j$ é a quantidade de \textit{jobs}, $o$ é a quantidade de operações e $m$ é a quantidade de máquinas. Para observar o comportamento do algoritmo em diversos cenários foram escolhidos problemas de tamanhos variados, sendo eles $[4, 12, 5]$, $[10, 29, 7]$, $[10, 30, 10]$ e $[15, 56, 10]$. Os problemas de teste podem ser encontrados nos apêndices do trabalho.\newline

        Os algoritmos foram executados no ambiente de nuvem Google Colab Code sendo executado no motor \textit{Python 3 Google Compute Engine Backend}. Para a execução foram utilizados somente processadores comuns (CPU) e 12 Gb de memória RAM.
    %% Fim do Interlúdio do [Metodologia de Pesquisa] %%

    %% 2 ::: Desenvolvimento
    %% 2.1 ::: Metodologia de Pesquisa
    %% 2.1.1
    \subsection{Objetivos}
        ... Analise visa descobrir e analisar como as alterações afetam o a qualidade do algoritmo...
    %% Fim do [Objetivos] %%
 %
%% Fim do [Metodologia de Pesquisa] %%


%% 2 ::: Desenvolvimento
%% 2.2
\section{Arquitetura}
    %% Interlúdio do [Arquitetura] %%
        A arquitetura de um projeto de é um fator que impacta diretamente no seu desempenho, porém como o objetivo desse trabalho é entender como o algoritmo PSO se comporta diante das alterações propostas, foi necessário o desenvolvimento de uma arquitetura que represente o problema de mais mais abstrata.
    %% Fim do Interlúdio do [Arquitetura] %%

    %% 2 ::: Desenvolvimento
    %% 2.2 ::: Arquitetura
    %% 2.2.1
    \subsection{Design do Projeto}
        Por esse trabalho ser desenvolvimento dentro de um grupo de estudos, foi obtado por usar uma estrutura de projeto mais didatica e de facil entendimento, de maneira que possa ser facilmente reutilizada por futuros estudantes. Porém essa escolha acarreta em uma certa perda de desempenho, porém, como se trata de um trabalho que visa explorar como cada tipo de alteração impacta o algoritmo, essa perca de desempenho não se torna um defeito.\newline
        
        O projeto foi desenvolvido na linguagem de programação Python versão 3.9.7 e utiliza além das bibliotecas basicas da linguagem, as biblotecas:
        \begin{itemize}
            \item NumPy (versão 1.21.3)
            \item MatPlotlib (versão 3.4.3)
        \end{itemize}

        \noindent O MatPlotlib foi utilizado para a geração de: 
        \begin{itemize}
            \item Graficos de Dispersão (\textit{Scatter Plot}) para representar a posição das particulas da população.
            \item Graficos de Superfice (\textit{Surface Plot}) para representar o espaço de soluções.
            \item Graficos de Barra Horizontal (\textit{Horizontal Bar Plot}) para representar um diagrama de Gantt com a solução do agendamento.
        \end{itemize}

        \noindent O NumPy foi utilizado para: 
        \begin{itemize}
            \item Grandes arranjos multi dimensionais de dados. 
            \item Funções randomicas de escolha.
            \item Ordenação de dados.
            \item Operações de calculo como raiz quadrada e potenciação.
            \item Funções de escolha de valores maximos e minimos.
            \item Modelagem, união e remodelagem de matrizes.
            \item Representações e calculo de vetores de movimentação.
        \end{itemize}
        


        %% Um mesmo espaço de soluções para todos os algoritmos.
        %% Diferentes espaçoes de soluções para as diferentes rodadas.

     %
    %% Fim do [Design do Projeto] %%


    %% 2 ::: Desenvolvimento
    %% 2.2 ::: Arquitetura
    %% 2.2.1
    \subsection{População}
        O algoritmo PSO também é bastante influenciado pela sua população inicial, mesmo nos algoritmos onde á mutação e consequentemente evolução dos indivíduos da população, as características como inercia, velocidade, direção e posição dessa população inicial são de grande importância. Assim, os algoritmos PSO utilizam a mesma função de geração de população inicial, representada pelo pseudo-código (Adicionar aqui).
    %% Fim do [População] %%


    %% 2 ::: Desenvolvimento
    %% 2.2 ::: Arquitetura
    %% 2.2.2
    \subsection{Espaço de Soluções}

        %% Mapa 2d x mapa 1d
        %% Mesh e interpolação de matrizes.
        %% Movimentação vetorial

        Como o algoritmo do PSO trabalha com um mapa de soluções, a maneira como ele é formado é de grande importância, por isso afim de tornar a comparação mais precisa, todos os casos de testes usam o mesmo algoritmo gerador do espaço de estados inicial, representado pelo pseudo-código (Adicionar aqui).
    %% Fim do [Espaço de Soluções] %%


    %% 2 ::: Desenvolvimento
    %% 2.2 ::: Arquitetura
    %% 2.2.3
    \subsection{Algoritmos}
        ...
    %% Fim do [Algoritmos] %%
    
 %
%% Fim do [Arquitetura] %%



%% 2 ::: Desenvolvimento
%% 2.3
\section{Execuções}
    Pela natureza variável do ambiente em nuvem cada teste de algoritmo foi executado 30 vezes para obter uma média dos tempos de execução. O teste de performance consiste em executar o algoritmo uma vez com cada problema de teste e salvar os dados de \textit{makespan}, tempo de execução e \textit{fitnes}.

    O teste de diversidade é feitos apenas entre as variações de PSO Dinâmico, afim de saber a taxa de mutação da população e como isso influenciou no \textit{makespan} e no \textit{fitnes} da solução.
%% Fim do [Execuções] %%



%% 2 ::: Desenvolvimento
%% 2.4
\section{Critérios de Avaliação}
    A partir das medias estatísticas das execuções, são utilizados como critérios de avaliação do algoritmo o tempo de \textit{makespan}, o \textit{fitnes} que é representado pelo diagrama de Gantt como pode ser visto no exemplo da figuraX. No caso das variações dinamicas do PSO a taxa de mutação da população.
    % Inserir diagrama de Gantt Aqui %
%% Fim do [Critérios de Avaliação] %%



%% 2 ::: Desenvolvimento
%% 2.5
\section{Resultados}
    (Ainda em desenvolvimento...)
    %\lipsum[1]
%% Fim do [Resultados] %%



%%%%%%%%%% Ir agr até a pagina 27 %%%%%%%%%%






% Esse captulo deve ir no minimo até a pagina 36