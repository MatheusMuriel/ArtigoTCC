%%%%%%%%%%%%%%%%%%%%%%%%%%%%%%%%%%%%%%%%%%%%%%%%%%%%%%%%%%%%%%%%%%%%%%%%%%%%%%%%%%%%%%%%%%%%%%%

%% 2 ::: Desenvolvimento
\chapter{Desenvolvimento}
    %% Interlúdio do [Desenvolvimento] %%
        ...
    %% Fim do Interlúdio do [Desenvolvimento] %%


%% 2 ::: Desenvolvimento
%% 2.1
\section{Metodologia de Pesquisa}
    %% Interlúdio do [Metodologia de Pesquisa] %%
        Para os testes deste trabalho estão sendo usados os mesmo problemas utilizados por \cite{Kacem2002}, e que já são largamente usados na literatura. E são representados por $[j, o, m]$ em que $j$ é a quantidade de \textit{jobs}, $o$ é a quantidade de operações e $m$ é a quantidade de máquinas. Para observar o comportamento do algoritmo em diversos cenários foram escolhidos problemas de tamanhos variados, sendo eles $[4, 12, 5]$, $[10, 29, 7]$, $[10, 30, 10]$ e $[15, 56, 10]$. Os problemas de teste podem ser encontrados nos apêndices do trabalho.\newline

        Os algoritmos foram executados no ambiente de nuvem Google Colab Code sendo executado no motor \textit{Python 3 Google Compute Engine Backend}. Para a execução foram utilizados somente processadores comuns (CPU) e 12 Gb de memória RAM.
    %% Fim do Interlúdio do [Metodologia de Pesquisa] %%

    %% 2 ::: Desenvolvimento
    %% 2.1 ::: Metodologia de Pesquisa
    %% 2.1.1
    \subsection{Objetivos}
        ... Analise visa descobrir e analisar como as alterações afetam o a qualidade do algoritmo...
    %% Fim do [Objetivos] %%
 %
%% Fim do [Metodologia de Pesquisa] %%


%% 2 ::: Desenvolvimento
%% 2.2
\section{Arquitetura}
    %% Interlúdio do [Arquitetura] %%
        A arquitetura de um projeto de é um fator que impacta diretamente no seu desempenho, porém como o objetivo desse trabalho é entender como o algoritmo PSO se comporta diante das alterações propostas, foi necessário o desenvolvimento de uma arquitetura que represente o problema de mais mais abstrata.
    %% Fim do Interlúdio do [Arquitetura] %%

    %% 2 ::: Desenvolvimento
    %% 2.2 ::: Arquitetura
    %% 2.2.1
    \subsection{Design do Projeto}
        Por esse trabalho ser desenvolvimento dentro de um grupo de estudos, foi obtado por usar uma estrutura de projeto mais didatica e de facil entendimento, de maneira que possa ser facilmente reutilizada por futuros estudantes. Porém essa escolha acarreta em uma certa perda de desempenho, porém, como se trata de um trabalho que visa explorar como cada tipo de alteração impacta o algoritmo, essa perca de desempenho não se torna um defeito.\newline
        
        O projeto foi desenvolvido na linguagem de programação Python versão 3.9.7 e utiliza além das bibliotecas basicas da linguagem, as biblotecas:
        \begin{itemize}
            \item NumPy (versão 1.21.3)
            \item MatPlotlib (versão 3.4.3)
        \end{itemize}

        \noindent O MatPlotlib foi utilizado para a geração de: 
        \begin{itemize}
            \item Graficos de Dispersão (\textit{Scatter Plot}) para representar a posição das particulas da população.
            \item Graficos de Superfice (\textit{Surface Plot}) para representar o espaço de soluções.
            \item Graficos de Barra Horizontal (\textit{Horizontal Bar Plot}) para representar um diagrama de Gantt com a solução do agendamento.
        \end{itemize}

        \noindent O NumPy foi utilizado para: 
        \begin{itemize}
            \item Grandes arranjos multi dimensionais de dados. 
            \item Funções randomicas de escolha.
            \item Ordenação de dados.
            \item Operações de calculo como raiz quadrada e potenciação.
            \item Funções de escolha de valores maximos e minimos.
            \item Modelagem, união e remodelagem de matrizes.
            \item Representações e calculo de vetores de movimentação.
        \end{itemize}
        
        Para trazer mais clareza ao codigo foi optado por criar algumas classes para a representação de entidades e abstração de logicas. Dentre elas estão:\newline
        
        Classe de Particula, que faz a representação de uma entidade de particula e armazena sua posição, velocidade e direção. Faz a inicialização da particula com valores aleatorios de posição, velocidade e direção. E estabelesse metodos para calcular o movimento da particula e o valor da qualidade de sua nova posição.\newline
        
        Classe de \textit{encoding}, responsavel por ler e interpretar os arquivos de instancia de um problema, criar uma representação para de solução, e então gerar o espaço de soluções com soluções geradas aleatoriamente.\newline
         
        Classe de \textit{decoding}, responsavel por decodificar uma solução no padrão de representação estabelecido pela classe de \textit{encoding} e calcular o \textit{fitness} dessa solução. Essa classe também estabelesse o metodo que gera o grafico de Gantt para representar visualmente a solução.\newline
        

        Como o PSO é um algoritmo estocástico, ou seja seu resultado é fruto de eventos não previsiveis, como a geração aleatoria de soluções, foi decidido que para obter uma media real dos desempenhos de cada algoritmo, em cada uma das 30 execuções será gerado um novo espaço de soluções e uma população inicial diferente.\newline
        E em cada uma dessas execuções será passado para cada um dos algoritmos uma \textit{Deepcopy} do espaço de soluções e da população inicial, para que os diferentes algoritmos trabalhem com base nos mesmos dados iniciais, para assim tem uma analise mais justa.\newline

        Para que a analise dos dados possa ser feita mais detalhadamente foram implementados alguns pontos de salvamento de dados, para salvar em arquivo texto as representações da população e do espaço de soluções gerados aleatoriamente, assim como a quantidade de rodadas que foram feitas, o historico da variavel $gBest$, o numero de mudanças de $pBest$, o historico de posição das particulas e as soluções finais encontradas.
     %
    %% Fim do [Design do Projeto] %%
    

    %% 2 ::: Desenvolvimento
    %% 2.2 ::: Arquitetura
    %% 2.2.2
    \subsection{Espaço de Soluções}
        O espaço de soluções (também chamado de espaço de busca) é um conceito que define um espaço imaginario aonde cada ponto desse espaço é uma possivel solução para o problema em questão. Um espaço de soluções pode ser representado de diversas maneiras, como listas e grafos, mas em problemas de optimização normalmente se usa uma representação na forma de uma matriz.\newline
        
        Nessa matriz de soluções cada localização $[i, j]$ tem uma solução valida para o problema, e o algoritmo em questão tem a tarefa de encontrar qual dessas posições tem a melhor solução, ou uma solução boa o suficiente. Na 
        \hyperref[fig:solution-space]{Figura \ref{fig:solution-space}} 
        é mostrado uma representação conceitual de um espaço de soluções em forma de matriz.\newline
        
        \begin{figure}[h]
            \centering
            \small{Incluir a imagem do exemplo de espaço de solução aqui}
            \caption{Figura X}
            \label{fig:solution-space}
        \end{figure}


        Algumas das implementações encontradas em bibliotecas de algoritmos populacionais de codigo aberto como \textit{PySwarms} utilizam internamente um sistema de representação de espaço de soluções de uma unica dimenção, ous eja, uma lista simples, e utiliza uma função para transformar uma posição $[x,y]$ em um indice da lista. Essa implementação foi escolhida por ser mais rapida e economica na maioria dos casos. \newline

        Como esse trabalho tem como foco uma exploração e implementação mais didatica foi optado a utilização de uma representação em forma de uma matriz com duas dimenções, utilizando a biblioteca \textit{NumPy}. Essa abordagem em matriz de duas dimenções também facilita a geração de graficos para representação visual, como é possivel ver na 
        \hyperref[fig:surfaceplot]{Figura \ref{fig:surfaceplot}} 
        xxx

        
        \begin{figure}[h]
            \centering
            \small{Incluir a grafico surface aqui}
            \caption{Grafico surface simples}
            \label{fig:surfaceplot}
        \end{figure}
        

        
        %% Mapa 2d x mapa 1d

        %% Influencia na solução

        %% Movimentação vetorial
        %% 

        E então depois de todas as analises acime e de modelagem foi desenvolvido para esse trabalho o algoritmo de inicialização populacional representado pelo 
        \hyperref[fig:pseudocodigo-solutionspace]{Pseudo codigo \ref{fig:pseudocodigo-solutionspace}} 
        mas que a implementação está no apêndice x desse trabalho.\hfill

        %% Todo - Apendice
        %% Todo - pseudo codigo
        \begin{figure}[h]
            \centering
            \small{Incluir o pseudo codigo de geração do espaço de soluções aqui}
            \caption{Pseudo codigo da geração do espaço de soluções}
            \label{fig:pseudocodigo-solutionspace}
        \end{figure}
     %
    %% Fim do [Espaço de Soluções] %%


    %% 2 ::: Desenvolvimento
    %% 2.2 ::: Arquitetura
    %% 2.2.3
    \subsection{Particula}
        ....
    %% Fim do [Particula] %%


    %% 2 ::: Desenvolvimento
    %% 2.2 ::: Arquitetura
    %% 2.2.4
    \subsection{População}
        O algoritmo PSO por ser um algoritmo paseado em inteligencia populacional, é muito influenciado pela geração de população inicial, mesmo nos algoritmos com variações aonde a mutação e consequentemente evolução dos indivíduos da população, as características como inercia, velocidade, direção e posição desses individuos da população inicial são de grande importância.\newline


        


        Assim, as variaçoes do algoritmo PSO implementadas nesse trabalho utilizam a mesma função geradora de população inicial, representada pelo pseudo-código   (Adicionar aqui).
    %% Fim do [População] %%


    %% 2 ::: Desenvolvimento
    %% 2.2 ::: Arquitetura
    %% 2.2.5
    \subsection{Algoritmos}
        ...
    %% Fim do [Algoritmos] %%
    
 %
%% Fim do [Arquitetura] %%



%% 2 ::: Desenvolvimento
%% 2.3
\section{Execuções}
    Pela natureza variável do ambiente em nuvem cada teste de algoritmo foi executado 30 vezes para obter uma média dos tempos de execução. O teste de performance consiste em executar o algoritmo uma vez com cada problema de teste e salvar os dados de \textit{makespan}, tempo de execução e \textit{fitnes}.

    O teste de diversidade é feitos apenas entre as variações de PSO Dinâmico, afim de saber a taxa de mutação da população e como isso influenciou no \textit{makespan} e no \textit{fitnes} da solução.
%% Fim do [Execuções] %%



%% 2 ::: Desenvolvimento
%% 2.4
\section{Critérios de Avaliação}
    A partir das medias estatísticas das execuções, são utilizados como critérios de avaliação do algoritmo o tempo de \textit{makespan}, o \textit{fitnes} que é representado pelo diagrama de Gantt como pode ser visto no exemplo da figuraX. No caso das variações dinamicas do PSO a taxa de mutação da população.
    % Inserir diagrama de Gantt Aqui %
%% Fim do [Critérios de Avaliação] %%



%% 2 ::: Desenvolvimento
%% 2.5
\section{Resultados}
    (Ainda em desenvolvimento...)
    %\lipsum[1]
%% Fim do [Resultados] %%



%%%%%%%%%% Ir agr até a pagina 31 %%%%%%%%%%






% Esse captulo deve ir no minimo até a pagina 36