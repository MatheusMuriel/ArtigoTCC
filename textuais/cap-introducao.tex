%% 1 ::: Introdução
\chapter{Introdução}

%%%%%%%%%%%%
Na sociedade contemporânea, a crescente demanda pelos mais variados tipos de produtos tem tornado necessário melhorias na produtividade das industrias, de modo que acompanhe o ritmo de crescimento da demanda.\\
%%%%%%
Segundo \citeauthor{Wari2016} essa necessidade acontece por causa de um ambiente altamente dinâmico no qual a demanda por um produto pode aumentar ou diminuir em um curto espaço de tempo, e a um cenário de grande concorrência entre as industrias, e que para atender a essas necessidades as industrias devem buscar meios de otimizar seus mecanismos de produção, e que essa melhoria é de suma importância para o sucesso dessas empresas.\\
%%%%%%%%%%%%

%%%%%%%%%%%%
Segundo \citeauthor{xhafa2008} uma das áreas de suma importância para a produtividade da linha de produção é a programação de produção. 
%%
A programação de produção é a parte da administração de produção responsável por decidir o local, o momento e a ordem nas quais serão realizadas as operações. 
%%
Por ser uma areá de suma importância e ter um escopo com variáveis e objetivos bem definidos a programação de produção se mostra como um bom ponto para se otimizar por meio de computação.\\
%%%%%%%%%%%%

%%%%%%%%%%%% Falar sobre problemas de Agendamento e FJSP
De acordo com a \citeauthor{Bagchi1999}, o problema de programação de produção se enquadra como um problema de escalonamento, que o autor define como um tipo de problema no qual existe um conjunto de demandas e um conjunto de recursos, e o objetivo é encontrar um agendamento que distribua as demandas entre os recursos de forma que o tempo de corrido entre o inicio e o fim da execução de todas as tarefas (chamado de \textit{makespan}) seja o menor possível.\\
%%%%%%%%%%%%

%%%%%%%%%%%% Definição formal do problema FJSP
\citeauthor{Bagchi1999} também define algumas sub divisões do problema de escalonamento, dentre elas o \textit{Job Shop Problem} \textit{(JSP)} e uma variação sua o \textit{Flexible Job Shop Problem} \textit{(FJSP)}. 
%%
Que é um problema no qual existem $m$ máquinas diferentes entre si e $n$ \textit{jobs} formados por diversas operações $O$ em uma ordem específica.\\
%%%%%%
Cada operação $O_{ij}$ tem um tempo de execução diferente em cada maquina, ou seja uma operação 
$O_{ij}$ demora um tempo $x$ se for executada na maquina $M_a$ e um tempo $y$ se for executa na maquina $M_b$.\\
Porém existem algumas restrições no problema de \textit{FJSP} como:
\begin{itemize}
    \item Não é possível executar simultaneamente duas operações de um mesmo \textit{job}.
    \item As maquinas são heterogêneas, ou sejam, não são iguais entre si.
    \item A execução de uma operação é atômica e não podem ser interrompidas.
    \item As máquinas são podem executar uma operação de cada vez.
    \item Um \textit{job} não pode ser processado duas vezes.
\end{itemize}

%%%%%%%%%%%%




%%%%%%%%%%%% Definição formal do problema FJSP
%%%%%%%%%%%% Falar sobre NP-Hard
%%%%%%%%%%%% Algoritmos bio inspirados
%%%%%%%%%%%% Algoritmos populacionais 
%%%%%%%%%%%% Algoritmos PSO
%%%%%%%%%%%% Problemas no PSO
%%%%%%%%%%%% Possiveis soluções
%%%%%%%%%%%% Oq esse trabalho faz
%%%%%%%%%%%% Objetivos e motivações
%%%%%%%%%%%%
Por causa dessa característica exponencial conforme o número de máquinas e o número de problemas, esse problema de escalonamento é classificado como um problema de otimização de análise combinatória e pertence à classe de problemas \textit{NP-Hard}.\\
%%%%%%%%%%%%
Os problemas da classe \textit{NP-Hard} são aqueles onde a resposta não pode ser encontrada computacionalmente em um tempo polinomial, ou seja, em um tempo razoável, porém uma solução pode ser verificada em tempo polinomial \cite{Eswaramurthy2008}.\\
%%%%%%%%%%%%
Existem diversos problemas clássicos de produção e manufatura enquadrados em problemas de otimização e escalonamento, o que torna esse assunto uma área de muito interesse para pesquisadores do mundo inteiro.\\
%%%%%%%%%%%%
Dentre os problemas clássicos de escalonamento e planejamento de produção estão: 
\textit{Single Machine Scheduling Problem}, 
\textit{Parallel Machine Scheduling Problem}, 
\textit{Flow Shop Scheduling Problem}, 
\textit{Job Shop Scheduling Problem} e 
\textit{Open Shop Scheduling Problem} 
\cite{Allahverdi2008}.\\
%%%%%%%%%%%%
Como esses problemas não são possíveis de serem resolvidos em tempo polinomial, não é possível encontrar uma solução perfeita para eles. Mas é possível encontrarmos uma solução boa o suficiente, conforme os critérios de avaliação do problema, essa solução é chamada solução ótima.\\
%%%%%%%%%%%%
\textit{Job Shop Scheduling Problem} (JSSP) é um do problema pertencente a classe de problemas \textit{NP-Hard}, ele chama muito a atenção de pesquisadores por ser um problema com diversas aplicações no mundo real, seja em ambientes de manufatura ou em planejamento de produção, ou até mesmo em logística \cite{Cheng1996}.\\
%%%%%%%%%%%%
Nesse problema, temos um conjunto $m$ de máquinas e um conjunto $n$ de tarefas chamadas \textit{jobs}, sendo cada \textit{job} uma sequência de operações, cada uma com seu determinado tempo de execução.\\ 
%%%%%%%%%%%%
O objetivo é encontrar um escalonamento que combine todas as máquinas de forma que minimize a quantidade de tempo ocioso de cada máquina, assim atingindo o objetivo de forma que seja mais econômica e eficiente \cite{Cheng1996}.\\
%%%%%%%%%%%%
O problema JSP é comprovadamente pertencente à classe \textit{NP-Hard} quando em um ambiente com duas ou mais máquinas, como demonstrado por \cite{Lenstra1979}.\\
%%%%%%%%%%%%
Porém, assim como visto por \cite{Bagchi1999} existem algumas restrições no cenário de JSP, dentre elas:
\begin{itemize}
    \item Duas operações do mesmo \textit{job} não podem ser executadas em simultâneo;
    \item Nenhuma máquina pode executar simultaneamente mais de uma operação;
    \item As restrições e configurações de processamento são conhecidas previamente e não são alteradas;
    \item Todo \textit{job} deve ser processado até o fim, mas é permitido que haja pausas e esperas entre suas operações;
    \item As máquinas são homogêneas;
    \item Nenhum \textit{job} pode ser processado duas vezes na mesma máquina;
    \item As operações são atômicas, não sendo possível interromper ou pausar a execução da mesma;
\end{itemize}
%%%%%%%%%%%%
O problema de \textit{Job Shop Flexível} ou \textit{Flexible Job Shop Problem} (FJSP) é uma extensão do JSP onde é possível que uma operação seja executada em mais de uma máquina. Sendo assim deve se além de determinar a ordem e o local de execução de cada \textit{job}, também é preciso determinada ordem e local de execução das operações. Assim sendo considerado uma extensão mais complexa do JSP \cite{Jansen2000}.\\
%%%%%%%%%%%%
Além disso, o problema de FJSP pode ser dividido em dois tipos, o parcial (P-FJSP), onde uma operação só pode ser executada por um subconjunto de máquinas, ou o total (T-FJSP), onde qualquer operação pode ser executada por qualquer máquina. O FJSP tem as mesmas restrições do JSP exceto a que diz que nenhum \textit{job} pode ser processado duas vezes na mesma máquina.\\
%%%%%%%%%%%%
Ao longo do tempo já foram propostas diversas abordagens para resolver o problema de FJSP, dentre elas o \textit{Branch and Bound} \cite{Nababan2008}, a 
\textit{Integer Programming} \cite{Pan2007}, a 
\textit{Dynamic Programming} \cite{Gromicho2012}, 
\textit{Evolutionary Algorithm} \cite{Pezzella2008}, e até mesmo técnicas híbridas \cite{Zhang2009}, onde duas ou mais abordagens são associadas para compor um algoritmo híbrido.\\
%%%%%%%%%%%%
Um desses algoritmos propostos é o algoritmo de Otimização por Enxame de Partículas ou \textit{Particle Swarm Optimization} proposto por \cite{Kennedy1995} e trabalha com um grupo de indivíduos cada um tendo: direção, velocidade, a informação da sua melhor posição e a informação da melhor posição entre todos os indivíduos do grupo, e com essas informações cada indivíduo consegue tirar sua média e a cada rodada do algoritmo ir chegando mais perto do objetivo.\\
%%%%%%%%%%%%
Outra abordagem que tem tido bons resultados em diversos problemas práticos \cite{Qing2012} são os Algoritmos Genéticos ou \textit{Genetic Algorithms} (GA), proposto por \textit{John Henry Holland} sendo inspirado na teoria da evolução de Charles Darwin, simulando a transmissão de genes dos indivíduos mais aptos por meio da simulação de operações de cruzamento e de mutações, para selecionar os indivíduos mais aptos.\\
%%%%%%%%%%%%
Neste trabalho iremos demonstrar e comparar a eficiência de uma abordagem híbrida do Algoritmo PSO com componentes evolutivos de GA, tornando assim esse algoritmo mais dinâmico, além de demonstrar sua eficiência em diferentes cenários reais de aplicação na indústria.

