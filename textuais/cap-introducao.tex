%% 1 ::: Introdução
\chapter{Introdução}
Em um ambiente de produção industrial moderna a otimização é um ponto de grande importância, devido à constante mudança e alta concorrência. Em ramos como a manufatura isso se mostra especialmente importante, podendo ser o fator decisório para o sucesso de uma empresa \cite{Wari2016}.\\
%%%%%%%%%%%%
\indent Um dos exemplos de otimização em sistemas de manufatura é dentro de um cenário onde existem diversas máquinas independentes e uma fila de tarefas não homogêneas, e o objetivo é achar uma programação de onde cada tarefa será executada e em qual ordem, de maneira a economizar o máximo de tempo e energia.\breakline
%%%%%%%%%%%%
Em um ambiente de alta concorrência é importante que essa ordem seja encontrada quanto antes, pois essa demora para encontrar a solução significa perda de tempo de produção. Porém, a busca de um bom escalonamento, ou seja, a configuração e ordem de execução, não é uma tarefa fácil. Pois, o número de possibilidades de arranjos cresce exponencialmente, e computar todas as soluções possíveis torna-se inviável após alguns níveis.\breakline
%%%%%%%%%%%%
    Por causa dessa característica exponencial conforme o número de máquinas e o número de problemas, esse problema de escalonamento é classificado como um problema de otimização de análise combinatória e pertence à classe de problemas \textit{NP-Hard}.\hfill\vspace{\onelineskip}

    Os problemas da classe \textit{NP-Hard} são aqueles onde a resposta não pode ser encontrada computacionalmente em um tempo polinomial, ou seja, em um tempo razoável, porém uma solução pode ser verificada em tempo polinomial \cite{Eswaramurthy2008}.\hfill\vspace{\onelineskip}

    Existem diversos problemas clássicos de produção e manufatura enquadrados em problemas de otimização e escalonamento, o que torna esse assunto uma área de muito interesse para pesquisadores do mundo inteiro.\hfill\vspace{\onelineskip}

    Dentre os problemas clássicos de escalonamento e planejamento de produção estão: 
    \textit{Single Machine Scheduling Problem}, 
    \textit{Parallel Machine Scheduling Problem}, 
    \textit{Flow Shop Scheduling Problem}, 
    \textit{Job Shop Scheduling Problem} e 
    \textit{Open Shop Scheduling Problem} 
    \cite{Allahverdi2008}.\hfill\vspace{\onelineskip}

    Como esses problemas não são possíveis de serem resolvidos em tempo polinomial, não é possível encontrar uma solução perfeita para eles. Mas é possível encontrarmos uma solução boa o suficiente, conforme os critérios de avaliação do problema, essa solução é chamada solução ótima.\hfill\vspace{\onelineskip}

    \textit{Job Shop Scheduling Problem} (JSSP) é um do problema pertencente a classe de problemas \textit{NP-Hard}, ele chama muito a atenção de pesquisadores por ser um problema com diversas aplicações no mundo real, seja em ambientes de manufatura ou em planejamento de produção, ou até mesmo em logística \cite{Cheng1996}.\hfill\vspace{\onelineskip}

    Nesse problema, temos um conjunto $m$ de máquinas e um conjunto $n$ de tarefas chamadas \textit{jobs}, sendo cada \textit{job} uma sequência de operações, cada uma com seu determinado tempo de execução.\hfill\vspace{\onelineskip} 

    O objetivo é encontrar um escalonamento que combine todas as máquinas de forma que minimize a quantidade de tempo ocioso de cada máquina, assim atingindo o objetivo de forma que seja mais econômica e eficiente \cite{Cheng1996}.\hfill\vspace{\onelineskip}

    O problema JSP é comprovadamente pertencente à classe \textit{NP-Hard} quando em um ambiente com duas ou mais máquinas, como demonstrado por \cite{Lenstra1979}.\hfill\vspace{\onelineskip}

    Porém, assim como visto por \cite{Bagchi1999} existem algumas restrições no cenário de JSP, dentre elas:
    \begin{itemize}
        \item Duas operações do mesmo \textit{job} não podem ser executadas em simultâneo;
        \item Nenhuma máquina pode executar simultaneamente mais de uma operação;
        \item As restrições e configurações de processamento são conhecidas previamente e não são alteradas;
        \item Todo \textit{job} deve ser processado até o fim, mas é permitido que haja pausas e esperas entre suas operações;
        \item As máquinas são homogêneas;
        \item Nenhum \textit{job} pode ser processado duas vezes na mesma máquina;
        \item As operações são atômicas, não sendo possível interromper ou pausar a execução da mesma;
    \end{itemize}

    O problema de \textit{Job Shop Flexível} ou \textit{Flexible Job Shop Problem} (FJSP) é uma extensão do JSP onde é possível que uma operação seja executada em mais de uma máquina. Sendo assim deve se além de determinar a ordem e o local de execução de cada \textit{job}, também é preciso determinada ordem e local de execução das operações. Assim sendo considerado uma extensão mais complexa do JSP \cite{Jansen2000}.\hfill\vspace{\onelineskip}

    Além disso, o problema de FJSP pode ser dividido em dois tipos, o parcial (P-FJSP), onde uma operação só pode ser executada por um subconjunto de máquinas, ou o total (T-FJSP), onde qualquer operação pode ser executada por qualquer máquina. O FJSP tem as mesmas restrições do JSP exceto a que diz que nenhum \textit{job} pode ser processado duas vezes na mesma máquina.\hfill\vspace{\onelineskip}

    Ao longo do tempo já foram propostas diversas abordagens para resolver o problema de FJSP, dentre elas o \textit{Branch and Bound} \cite{Nababan2008}, a 
    \textit{Integer Programming} \cite{Pan2007}, a 
    \textit{Dynamic Programming} \cite{Gromicho2012}, 
    \textit{Evolutionary Algorithm} \cite{Pezzella2008}, e até mesmo técnicas híbridas \cite{Zhang2009}, onde duas ou mais abordagens são associadas para compor um algoritmo híbrido.\hfill\vspace{\onelineskip}

    Um desses algoritmos propostos é o algoritmo de Otimização por Enxame de Partículas ou \textit{Particle Swarm Optimization} proposto por \cite{Kennedy1995} e trabalha com um grupo de indivíduos cada um tendo: direção, velocidade, a informação da sua melhor posição e a informação da melhor posição entre todos os indivíduos do grupo, e com essas informações cada indivíduo consegue tirar sua média e a cada rodada do algoritmo ir chegando mais perto do objetivo.\hfill\vspace{\onelineskip}

    Outra abordagem que tem tido bons resultados em diversos problemas práticos \cite{Qing2012} são os Algoritmos Genéticos ou \textit{Genetic Algorithms} (GA), proposto por \textit{John Henry Holland} sendo inspirado na teoria da evolução de Charles Darwin, simulando a transmissão de genes dos indivíduos mais aptos por meio da simulação de operações de cruzamento e de mutações, para selecionar os indivíduos mais aptos.\hfill\vspace{\onelineskip}

    Neste trabalho iremos demonstrar e comparar a eficiência de uma abordagem híbrida do Algoritmo PSO com componentes evolutivos de GA, tornando assim esse algoritmo mais dinâmico, além de demonstrar sua eficiência em diferentes cenários reais de aplicação na indústria.\hfill

