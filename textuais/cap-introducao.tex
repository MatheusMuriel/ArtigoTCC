%% 1 ::: Introdução
\chapter{Introdução}

%%%%%%%%%%%%
Na sociedade contemporânea, a crescente demanda pelas mais variadas categorias de produtos tem tornado necessário melhorias na produtividade das indústrias, de modo que acompanhe o ritmo de crescimento da demanda.
%%%%%%%%%%%%


%%%%%%%%%%%%
Essa necessidade acontece devido a um ambiente altamente dinâmico onde a demanda por um produto pode aumentar ou diminuir em um curto espaço de tempo, e a um cenário de grande concorrência entre as indústrias, e que para atender a essas necessidades as indústrias devem buscar meios de otimizar seus mecanismos de produção, e que essa melhoria é importante para o sucesso dessas empresas \cite{Wari2016}.
%%%%%%%%%%%%


%%%%%%%%%%%%
\indent Segundo \citeauthor{Xhafa2008} uma dessas áreas importantes para a produtividade da linha de produção é a programação de produção. 
A programação de produção é a parte da administração de produção responsável por decidir o local, o momento e a ordem nas quais serão realizadas as operações. 
%%
Por ser uma areá importante e ter um escopo com variáveis e objetivos bem definidos a programação de produção se mostra como um bom ponto para se otimizar por meio da computação.\\
%%%%%%%%%%%%
%%%%%%%%%%%% Falar sobre problemas de Agendamento e FJSP
\indent De acordo com a \citeauthor{Bagchi1999}, o problema de programação de produção se enquadra como um problema de escalonamento, que o autor define como um tipo de problema onde existe um conjunto de demandas e um conjunto de recursos, e o objetivo é encontrar um agendamento que distribua as demandas entre os recursos de forma que o tempo de corrido entre o início e o fim da execução de todas as tarefas (chamado \textit{makespan}) seja o menor possível.\\
%%%%%%%%%%%%
%%%%%%%%%%%% Definição formal do problema FJSP
\indent \citeauthor{Bagchi1999} também define algumas sub divisões do problema de escalonamento, dentre elas o \gls{jsp} e uma variação sua o \gls{fjsp}. 
%%
Que é um problema onde existem $m$ máquinas diferentes entre si e $n$ \textit{jobs} formados por diversas operações $O$ em uma ordem específica.\\
%%%%%%
Cada operação $O_{ij}$ tem um tempo de execução diferente em cada máquina, ou seja, uma operação 
$O_{ij}$ demora um tempo $x$ se for executada na máquina $M_a$ e um tempo $y$ se for executa na maquina $M_b$.\\
Porém, existem algumas restrições no problema de \textit{FJSP} como:
\begin{itemize}
    \item Não é possível executar simultaneamente duas operações de um mesmo \textit{job}.
    \item As maquinas são heterogêneas, ou sejam, não são iguais entre si.
    \item A execução de uma operação é atômica e não podem ser interrompidas.
    \item As máquinas são podem executar uma operação de cada vez.
    \item Um \textit{job} não pode ser processado duas vezes.
\end{itemize}
%%%%%%%%%%%%

%%%%%%%%%%%% Falar sobre NP-Hard
O número de possibilidades de arranjos pode ser representado por: $(n!)^m$, ou seja, o número de soluções possíveis cresce exponencialmente conforme o número de máquinas e de \textit{jobs}.
%%%%
Esse característica de crescimento exponencial faz com que um problema se torne inviável de ser resolvido de formas tradicionais após um certo tamanho, pois é necessário computar todas as suas possibilidades de soluções.
%%%%
Isso faz com que o \textit{FJSP} seja definido como um problema de otimização com crescimento exponencial, o que de acordo com \citeauthor{Eswaramurthy2008} o define como um problema pertencente a classe de problemas \textit{NP-Hard}.\hfill

%%%%
\indent A classe de problemas \textit{NP-Hard} é composta por problemas cujo a resposta não pode ser encontrada computacionalmente em um tempo polinomial, ou seja, em um tempo aceitável, porém uma solução pode ser verificada em um tempo polinomial \cite{Eswaramurthy2008}.
%%%%%%%%%%%%
%%%%%%%%%%%% Algoritmos bio inspirados
Para solucionar problemas da classe \textit{NP-Hard} são geralmente utilizados métodos de aproximação, que visam ao invés de tentar encontrar a melhor solução possível, tentar encontrar uma solução boa o suficiente para o problema em questão.
%%
Um dos algoritmos que fazem essa busca por uma solução via aproximação são os algoritmos bio inspirados, que se baseiam técnicas observadas na natureza.\\
%%
\indent A ideia de se basear em um comportamento biológico visa usar técnicas que passaram pela seleção natural ao longo de milhares de anos, e que como resultado dessa seleção sobraram somente as técnicas mais eficientes.
%%%%%%%%%%%%
%%%%%%%%%%%% Algoritmos populacionais 
Uma subcategoria dos algoritmos bioinspirados são os algoritmos populacionais, baseados no comportamento de populações de animais como aves, abelhas e insetos.
%%
Esses algoritmos visam utilizar o conceito de inteligência de bando, simulando indivíduos interagindo entre si e com o ambiente para chegar a um objetivo.
%%
Existem diferentes algoritmos populacionais como o \textit{Ant Colony Optimization}, o \textit{Stochastic Diffusion Search} e o \textit{Particle Swarm Optimization}.
%%%%%%%%%%%%
%%%%%%%%%%%% Algoritmos PSO
\section{Particle Swarm Optimization (PSO)}
%\indent O algoritmo \textit{Particle Swarm Optimization} \textit{(PSO)} 
\indent O algoritmo \gls{pso}, ou Optimização por Enxame de Partículas em tradução para o português, se baseia na convergência de um enxame de partículas em um objetivo.
%%
A estrutura básica de um algoritmo \textit{PSO} é uma população de partículas em que cada partícula tem uma velocidade e uma direção, além das informações da sua posição, o quão boa sua posição atual é, qual foi a melhor posição na, qual ela já esteve ($pBest$), e qual foi a melhor posição em que qualquer um dos indivíduos da população já esteve ($gBest$).
%%
A cada rodada as partículas se movimentam para uma direção intermediaria entre $pBest$ e $gBest$, por uma distância que é calculada pelo valor de velocidade menos o valor de inércia da partícula multiplicado por um valor aleatório entre $0.1$ e $0.9$. 
E assim a cada rodada do algoritmo a população vai chegando cada vez mais perto da melhor solução.\\
%%%%%%%%%%%%
%%%%%%%%%%%% Problemas no PSO
\indent Alguns dos problemas que podem acontecer no \textit{PSO} é uma convergência prematura da população em mínimos locais,
%%
que são casos aonde entre as soluções existe uma que é melhor que a média das soluções a sua volta, porém não é o melhor entre todas.\\
%%
Como a movimentação das partículas se baseia na média entre as melhores posições locais e gerais do grupo, caso a maioria das partículas vá para em direção ao mínimo local o $gBest$ fica no mínimo local, e as partículas tentem a não saírem dessa região.\\
%%%%%%%%%%%%
%%%%%%%%%%%% Possíveis soluções
\indent Algumas medidas podem ser tomadas para diminuir essas possibilidades de convergência Prematura.
%%
Como as variáveis de inércia e de velocidade são variáveis individuais que influenciam a movimentação da partícula, é possível utilizar técnicas como valores dinâmicos e pesos relativos para as variáveis, e assim mudar de maneira dinâmica as características de movimentação das partículas.\\
%%%%%%%%%%%%
%%%%%%%%%%%% Oq esse trabalho faz
\indent Nesse trabalho são analisados os fatores de projeto, implementação e a utilização de uma abordagem dinâmicas nas variáveis de partícula e como essas modificações influenciam na convergência do \textit{PSO} aplicado em cenários de aplicação para a solução de problemas do tipo \textit{FJSP}.
%%%%%%%%%%%%

%%%%%%%%%%%% Objetivos e motivações
%%%%%%%%%%%%

(Oq foi feito.....)

Atualmente existem poucos estudos sobre o comportamento do PSO com componentes dinâmicos em cenários mono objetivos, principalmente em problemas com o FJSP. 
%
Com as implementações feitas nesse trabalho se espera não uma análise crua de números e métricas de desempenho, mas sim ideias de como certas alterações afetam o desempenho do algoritmo, e quais são as tendências de mudanças.