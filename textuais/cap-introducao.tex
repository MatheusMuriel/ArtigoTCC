%% 1 ::: Introdução
\chapter{Introdução}

%%%%%%%%%%%%
Na sociedade contemporânea, a crescente demanda pelos mais variados tipos de produtos tem tornado necessário melhorias na produtividade das industrias, de modo que acompanhe o ritmo de crescimento da demanda.\\
%%%%%%
Segundo \citeauthor{Wari2016} essa necessidade acontece por causa de um ambiente altamente dinâmico no qual a demanda por um produto pode aumentar ou diminuir em um curto espaço de tempo, e a um cenário de grande concorrência entre as industrias, e que para atender a essas necessidades as industrias devem buscar meios de otimizar seus mecanismos de produção, e que essa melhoria é de suma importância para o sucesso dessas empresas.\\
%%%%%%%%%%%%

%%%%%%%%%%%%
Segundo \citeauthor{Xhafa2008} uma das áreas de suma importância para a produtividade da linha de produção é a programação de produção. 
%%
A programação de produção é a parte da administração de produção responsável por decidir o local, o momento e a ordem nas quais serão realizadas as operações. 
%%
Por ser uma areá de suma importância e ter um escopo com variáveis e objetivos bem definidos a programação de produção se mostra como um bom ponto para se otimizar por meio de computação.\\
%%%%%%%%%%%%

%%%%%%%%%%%% Falar sobre problemas de Agendamento e FJSP
De acordo com a \citeauthor{Bagchi1999}, o problema de programação de produção se enquadra como um problema de escalonamento, que o autor define como um tipo de problema no qual existe um conjunto de demandas e um conjunto de recursos, e o objetivo é encontrar um agendamento que distribua as demandas entre os recursos de forma que o tempo de corrido entre o inicio e o fim da execução de todas as tarefas (chamado de \textit{makespan}) seja o menor possível.\\
%%%%%%%%%%%%

%%%%%%%%%%%% Definição formal do problema FJSP
\citeauthor{Bagchi1999} também define algumas sub divisões do problema de escalonamento, dentre elas o \textit{Job Shop Problem} \textit{(JSP)} e uma variação sua o \textit{Flexible Job Shop Problem} \textit{(FJSP)}. 
%%
Que é um problema no qual existem $m$ máquinas diferentes entre si e $n$ \textit{jobs} formados por diversas operações $O$ em uma ordem específica.\\
%%%%%%
Cada operação $O_{ij}$ tem um tempo de execução diferente em cada maquina, ou seja uma operação 
$O_{ij}$ demora um tempo $x$ se for executada na maquina $M_a$ e um tempo $y$ se for executa na maquina $M_b$.\\
Porém existem algumas restrições no problema de \textit{FJSP} como:
\begin{itemize}
    \item Não é possível executar simultaneamente duas operações de um mesmo \textit{job}.
    \item As maquinas são heterogêneas, ou sejam, não são iguais entre si.
    \item A execução de uma operação é atômica e não podem ser interrompidas.
    \item As máquinas são podem executar uma operação de cada vez.
    \item Um \textit{job} não pode ser processado duas vezes.
\end{itemize}
%%%%%%%%%%%%

%%%%%%%%%%%% Falar sobre NP-Hard
O numero de possibilidades de arranjos é $(n!)^m$, ou seja, o numero de soluções possíveis cresce exponencialmente de acordo com o numero de maquinas e de \textit{jobs}.\\
%%%%
Esse característica de crescimento exponencial faz com que um problema se torne inviável de ser resolvido de formas tradicionais após um certo tamanho, pois é necessário computar todas as suas possibilidades de soluções.\\
%%%%
Isso faz com que o \textit{FJSP} seja definido como um problema de otimização com crescimento exponencial, o que de acordo com \citeauthor{Eswaramurthy2008} o define como um problema pertencente a classe de problemas \textit{NP-Hard}.\\
%%%%
A classe de problemas \textit{NP-Hard} é composta por problemas cujo a resposta não pode ser encontrada computacionalmente em um tempo polinomial, ou seja, em um tempo aceitável, porém uma solução pode ser verificada em um tempo polinomial \cite{Eswaramurthy2008}.\\
%%%%%%%%%%%%


%%%%%%%%%%%% Algoritmos bio inspirados
Para solucionar problemas da classe \textit{NP-Hard} geralmente são utilizados métodos de aproximação, que tem como objetivo ao invés de tentar encontrar a melhor solução possível, tentar encontrar uma solução boa o suficiente para o problema em questão.
%%
Um dos tipos de algoritmos que fazem essa busca de uma solução via aproximação são os algoritmos bio inspirados, que se baseiam técnicas observadas na natureza.\\
%%
\indent A ideia de se basear em um comportamento biológico tem como objetivo usar técnicas que passaram pela seleção natural ao longo de milhares de anos, e que como resultado dessa seleção sobraram somente as técnicas mais eficientes.\\
%%%%%%%%%%%%

%%%%%%%%%%%% Algoritmos populacionais 
Uma sub categoria dos algoritmos bioinspirados são os algoritmos populacionais, baseados no comportamento de populações de animais como aves, abelhas e insetos.
%%
Esses algoritmos visam utilizar o conceito de inteligência de bando. Simulando indivíduos interagindo entre si e com o ambiente para encontrar um objetivo.\\
%%
Existem diferentes algoritmos populacionais como o \textit{Ant Colony Optimization}, o \textit{Stochastic Diffusion Search} e o \textit{Particle Swarm Optimization}.\\
%%%%%%%%%%%%

%%%%%%%%%%%% Algoritmos PSO
O algoritmo \textit{Particle Swarm Optimization} \textit{(PSO)} ou Optimização por Enxame de Partículas se baseia na convergencia de um enxame de particulas em um objetivo.\\
%%
A estrutura básica de um algoritmo \textit{PSO} é uma população de partículas em que cada partícula tem uma velocidade e uma direção, além das informações da sua posição, o quão boa sua posição atual é, qual foi a melhor posição na qual ela já esteve ($pBest$), e qual foi a melhor posição em que qualquer um dos individos da população já esteve ($gBest$).\\
%%
E assim a cada rodada as partículas se movimentam para uma média entre ($pBest$ e $gBest$) e assim a cada rodada do algoritmo a população vai chegando cada vez mais perto da melhor solução.
%%%%%%%%%%%%

%%%%%%%%%%%% Problemas no PSO
%%%%%%%%%%%%

%%%%%%%%%%%% Possíveis soluções
%%%%%%%%%%%%

%%%%%%%%%%%% Oq esse trabalho faz
%%%%%%%%%%%%

%%%%%%%%%%%% Objetivos e motivações
%%%%%%%%%%%%

