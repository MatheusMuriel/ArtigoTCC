\chapter{Desenvolvimento}

\section{Metodologia de Pesquisa}

Para os testes deste trabalho estão sendo usados os mesmo problemas utilizados por \cite{Kacem2002}, e que já são largamente usados na literatura. E são representados por $[j, o, m]$ em que $j$ é a quantidade de \textit{jobs}, $o$ é a quantidade de operações e $m$ é a quantidade de máquinas. Para observar o comportamento do algoritmo em diversos cenários foram escolhidos problemas de tamanhos variados, sendo eles $[4, 12, 5]$, $[10, 29, 7]$, $[10, 30, 10]$ e $[15, 56, 10]$. Os problemas de teste podem ser encontrados nos apêndices do trabalho.
\newline

Os algoritmos foram executados no ambiente de nuvem Google Colab Code sendo executado no motor \textit{Python 3 Google Compute Engine Backend}. Para a execução foram utilizados somente processadores comuns (CPU) e 12 Gb de memória RAM.

\section{Execuções}
Pela natureza variável do ambiente em nuvem cada teste de algoritmo foi executado 30 vezes para obter uma média dos tempos de execução. O teste de performance consiste em executar o algoritmo uma vez com cada problema de teste e salvar os dados de \textit{makespan}, tempo de execução e \textit{fitnes}.

O teste de diversidade é feitos apenas entre as variações de PSO Dinâmico, afim de saber a taxa de mutação da população e como isso influenciou no \textit{makespan} e no \textit{fitnes} da solução.

\section{Critérios de Avaliação}
A partir das medias estatísticas das execuções, são utilizados como critérios de avaliação do algoritmo o tempo de \textit{makespan}, o \textit{fitnes} que é representado pelo diagrama de Gantt como pode ser visto no exemplo da figuraX. No caso das variações dinamicas do PSO a taxa de mutação da população.
% Inserir diagrama de Gantt Aqui %

\section{Iniciação do espaço de soluções}
Como o algoritmo do PSO trabalha com um mapa de soluções, a maneira como ele é formado é de grande importância, por isso afim de tornar a comparação mais precisa, todos os casos de testes usam o mesmo algoritmo gerador do espaço de estados inicial, representado pelo pseudo-código (Adicionar aqui).

\section{Iniciação da população}
O algoritmo PSO também é bastante influenciado pela sua população inicial, mesmo nos algoritmos onde á mutação e consequentemente evolução dos indivíduos da população, as características como inercia, velocidade, direção e posição dessa população inicial são de grande importância. Assim, os algoritmos PSO utilizam a mesma função de geração de população inicial, representada pelo pseudo-código (Adicionar aqui).

\section{Resultados}
(Ainda em desenvolvimento...)
\lipsum[1]