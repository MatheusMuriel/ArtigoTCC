\chapter{Introdução}
%% Introdução ao que vai ser o trabalho %%
Em um ambiente de produção industrial moderna a optimização é um ponto de grande importância, devido a constante mudança e alta concorrência. Em ramos como a manufatura isso se mostra especialmente importante, podendo ser o fator decisório para o sucesso de uma empresa  (WARI; ZHU, 2016).\newline

Um dos exemplos de otimização dentro de sistemas de manufatura é dentro de um cenário com diversas máquinas independentes e uma fila de tarefas não homogêneas, achar uma programação de onde cada tarefa será executada e em qual ordem, de maneira a economizar tempo e energia.\newline

Em um ambiente de alta concorrência é importante que essa ordem seja encontrada o quanto antes, pois demora para encontrar a solução significa perda de tempo de produção. Porém a busca de um bom escalonamento, ou seja, a configuração e ordem de execução, não é uma tarefa fácil. Pois o número de possibilidades de arranjos cresce exponencialmente, e computar todas as soluções possíveis torna-se inviável após alguns níveis.\newline

Por causa dessa característica exponencial de acordo com o número de máquinas e o número de problemas, esse problema de escalonamento é classificado como um problema de optimização de análise combinatória e pertence a classe de problemas NP-Hard. Segundo Eswaramurthy (2008), problemas NP-Hard são aqueles em que a resposta não pode ser encontrada computacionalmente em tempo polinomial, ou seja, um tempo razoável, porém uma solução pode ser verificada em tempo polinomial, um outro exemplo de problema NP-Hard é encontrar um número primo, que é muito utilizado em sistemas de criptografia.\newline

Existem diversos problemas clássicos de produção e manufatura que são enquadrados em problemas de otimização e escalonamento, o que torna esse assunto uma área de muito interesse para pesquisadores do mundo inteiro.\newline

Dentre os clássicos de escalonamento e planejamento de produção estão: Single Machine Scheduling Problem, Parallel Machine Scheduling Problem, Flow Shop Scheduling Problem, Job Shop Scheduling Problem e Open Shop Scheduling Problem (ALLAHVERDI et al., 2008).\newline

Como esses problemas não são possíveis de serem resolvidos em tempo polinomial, não é possível encontrar uma solução para eles. Mas é possível encontrarmos uma solução “boa o suficiente”, que chamamos de solução ótima. Para encontrar uma solução ótima para um problema de análise escalonamento HP-Hard existem diversas abordagens.\newline
%% Fim Introdução %%



\section{Problema de Job Shop Flexível - FJSP}
%% Oque é FJSP %%
Para esse trabalho foi escolhido o cenário de Flexible Job Shop  Scheduling Problem (FJSSP), que é uma variação do Job Shop  Scheduling Problem (JSSP).

\subsection{Problema de Job Shop - JSP} 
%% Oque é o problema de JSP Simples %%
Job Shop Scheduling Problem (JSSP) é um problema clássico, pertencente a classe de problemas NP-Hard, ele chama muito a atenção de pesquisadores por ser uma problema com muitas aplicações no mundo real, seja em ambientes de manufatura ou em planejamento de produção ou até mesmo em logística. [Cheng et al.1996].\newline

Nesse problema, temos um conjunto de máquinas e um conjunto de tarefas (os jobs), sendo cada job uma sequência de operações, e o objetivo é encontrar uma sequência que combine as máquinas para minimize a quantidade de tempo ocioso de cada máquina, assim atingindo um ou mais objetivos. [Cheng et al.1996]. O problema de JSP é comprovadamente pertencente à classe NP-Hard como demonstrado por [Lenstra e Rinnooy 1979] quando existem duas ou mais máquinas. \newline

Ao longo do tempo já foram propostas diversas abordagens para resolver o problema de JSP, dentre elas o branch and bound [Nababan et al. 2008], a Programação Inteira [Pan 1997], a Programação Dinâmica [Gro-micho et al. 2012], Algoritmos Evolutivos [Pezzella et al.2008], e até mesmo técnicas híbridas [Zhang et al. 2009], onde duas ou mais abordagens são associadas para compor um algoritmo híbrido.\newline
%% Fim do JSP %%

\subsubsection{Problema de Job Shop Flexível - FJSP}
%% Diferenças para o problema de Job Shop Flexível %%
\lipsum[3]

\subsubsection{Cenários de Aplicação}
%% Onde da para usar e onde resolve problemas %%
\lipsum[3]

\section{Soluções Existentes}
%% Estado da arte e algoritmos que solucionam %%
\lipsum[3]

\subsection{Algoritmos Genéticos - GA}
%% Falar sobre GA %%
\lipsum[2]

%% TODO - Adicionar outras soluções existentes %%

%% Citar aqui no final o PSO %%

\section{Particle Swarm Optimization - PSO}
\lipsum[4]
%% Falar mais a fundo sobre o PSO %%

\subsection{Historia}
\lipsum[2]
%% Historia do PSO %%

\subsection{Aplicações}
\lipsum[3]
%% Aplicações do PSO %% 

\subsection{Defeitos e Problemas}
\lipsum[2]
%% Defeitos do PSO %%

\section{Melhorias do Algoritmo PSO}
\lipsum[2]
%% Falar de abordagens e melhorias no PSO %%

\subsection{Multithreading}
\lipsum[2]
%% Ambientes distribuídos %%

\subsection{Hiper Heurísticos}
\lipsum[2]
%% Utilização de Hiper Heurísticas para melhorar o PSO %%

\subsection{Hibridização}
\lipsum[2]
%% Sobre a possibilidade de usar GA no PSO %% 

%% TPSO, DIPSO, etc... %%




