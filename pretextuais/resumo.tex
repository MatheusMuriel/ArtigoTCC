%%Altere as informações do resumo%%
\noindent{
    MURIEL; MATHEUS, F. 
    \textbf{\imprimirtitulo}. 
    Trabalho de Conclusão de Curso (Graduação) - 
    \imprimirinstituicao. 
    \imprimirlocal, 
    \imprimirdata.
}
\par

\begin{resumo}
    O \textit{Flexible Job Shop Problem} (FJSP) é um complexo problema de otimização de análise combinatória e pertence à classe de problemas NP-Hard. Esse problema consiste em um cenário onde existem diversos conjuntos de tarefas, chamadas \textit{jobs}, e diversas maquinas diferentes que podem processar essa tarefa, e tem como objetivo encontrar um agendamento que execute todas as operações na menor quantidade de tempo possível. Por se tratar de um problema NP-Hard não é possível encontrar a melhor solução, então ao longo do tempo foram propostos diversos tipos de algoritmos para encontrar uma solução suficiente boa para esse problema, uma dessas abordagens de algoritmos são os algoritmos bio inspirados nos quais se utilizam padrões observados na natureza como heurísticas para resolver problemas. Um desses padrões observados na natureza é o comportamento de enxames de abelhas e o movimento migratório de bandos de aves, o comportamento desses bandos é usado como heurística no algoritmo \textit{Particle Swarm Optimization} (PSO) onde cada individuo de uma população de partículas utiliza a média entre sua melhor posição já visitada (\textit{pBest}) e a melhor posição já visitada entre todos da população (\textit{gBest}) para obter uma nova posição, assim a cada rodada a população converge para um resultado ótimo. Porém, em alguns casos o algoritmo PSO pode convergir prematuramente para mínimos locais e convergir para uma solução não ótima. Já foram propostas modificações dinâmicas no PSO para resolver o problema de FJSP, mas apenas em cenários multiobjetivo, ou seja, com mais de um critério de avaliação de qualidade. Esse trabalho descreve a implementação, testes e resultados de uma nova abordagem introduzindo componentes de inércia dinâmicos nas partículas e obteve resultados promissores que demonstram uma significativa melhora na utilização dessa nova abordagem.\vspace{\onelineskip}

%%Adicione as palavras chaves após os dois pontos '':''%%
%\noindent\textbf{Palavras-chaves}: 3 ou mais.
\noindent\textbf{Palavras-chave}: PSO; FJSP; Inteligencia Populacional; Algoritmos de partículas; Particle Swarm Optimization; Flexible Job Shop Problem.
\end{resumo}
