%%Altere as informações do resumo%%
\noindent{
    MURIEL; MATHEUS, F. 
    \textbf{\imprimirtitulo}. 
    Trabalho de Conclusão de Curso (Graduação) - 
    \imprimirinstituicao. 
    \imprimirlocal, 
    \imprimirdata.
}
\par

\begin{resumo}
    O \textit{Flexible Job Shop Problem} (FJSP) é um complexo problema de otimização de análise combinatória pertencente à classe de problemas NP-Hard. Esse problema consiste em um cenário onde existem diversos conjuntos de tarefas, chamadas \textit{jobs}, e diversas máquinas diferentes que podem processar essa tarefa, e tem como objetivo encontrar um agendamento que execute todas as operações na menor quantidade de tempo possível. Por se tratar de um problema NP-Hard não é possível encontrar a melhor solução, então ao longo do tempo foram propostos diversos tipos de algoritmos para encontrar uma solução boa o suficiente para o problema, uma dessas abordagens são os algoritmos bio inspirados nos quais se utilizam padrões observados na natureza para resolver problemas. Um desses padrões observados na natureza é o comportamento de enxames de abelhas e o movimento migratório de bandos de aves, o comportamento desses bandos é usado como heurística no algoritmo \textit{Particle Swarm Optimization} (PSO) onde cada individuo de uma população de partículas utiliza a média entre sua melhor posição já visitada (\textit{pBest}) e a melhor posição já visitada entre todos da população (\textit{gBest}) para obter uma nova posição, assim a cada rodada a população converge para um resultado ótimo. Porém, em alguns casos o algoritmo PSO pode convergir prematuramente para um mínimo local e chegar a uma solução não ótima. Já foram propostas modificações dinâmicas no PSO para resolver o problema de FJSP, mas apenas em cenários multiobjetivo, ou seja, com mais de um critério de avaliação de qualidade. Esse trabalho foca no cenário mono objetivo e descreve a implementação, testes e resultados de uma nova abordagem, introduzindo componentes de inércia dinâmicos nas partículas e obteve resultados que demonstram uma redução de $3,90\%$ no tempo final de \textit{makespan}, de $16,63\%$ no desvio padrão e de $33,76\%$ na variância. Uma significativa melhora na confiabilidade e na chance de se obter melhores resultados na execução do algoritmo em comparação com PSO base.\vspace{\onelineskip}

%%Adicione as palavras chaves após os dois pontos '':''%%
%\noindent\textbf{Palavras-chaves}: 3 ou mais.
\noindent\textbf{Palavras-chave}: PSO; Otimização por enxame de partículas; Particle Swarm Optimization; FJSP; Flexible Job Shop Problem; Inteligência Populacional; Componentes dinâmicos.
\end{resumo}
