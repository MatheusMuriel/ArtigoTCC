%%Altere as informações do resumo%%
\noindent{
  SOBRENOME; NOME, D. 
  \textbf{\imprimirtitulo}. 
  Trabalho de conclusão de curso (Graduação) - 
  \imprimirinstituicao. 
  \imprimirlocal, 
  \imprimirdata.
}
\par

%%Se o seu resumo não for em inglês, altere o ``Abstract'' e ``english'' abaixo.
\begin{resumo}[Abstract]
  \begin{otherlanguage*}{english}
    %%Write your abstract in foreign language here%%
    \emph{
      The \textit{Flexible Job Shop Problem} (FJSP) is a complex combinatorial analysis optimization problem belonging to the class of NP-Hard problems. This problem consists of a scenario where there are several sets of tasks, called \textit{jobs}, and several different machines that can process this task, and its objective is to find a schedule that performs all operations in the shortest amount of time possible. As it is an NP-Hard problem, it is not possible to find the best solution, so over time several types of algorithms have been proposed to find a good enough solution to the problem, one of these approaches is the bio inspired algorithms in which they use patterns observed in nature to solve problems. One of these patterns observed in nature is the behavior of bee swarms and the migratory movement of flocks of birds, the behavior of these flocks is used as a heuristic in the \textit{Particle Swarm Optimization} (PSO) algorithm where each individual of a population of particles uses the average between its best position ever visited (\textit{pBest}) and the best position already visited among all of the population (\textit{gBest}) to obtain a new position, so that each round the population converges to an optimal result . However, in some cases the PSO algorithm may prematurely converge to a local minimum and arrive at a non-optimal solution. Dynamic modifications in the PSO have already been proposed to solve the FJSP problem, but only in multi-objective scenarios, that is, with more than one quality assessment criterion. This work focuses on the single objective scenario and describes the implementation, tests and results of a new approach, introducing dynamic inertial components into particles and obtained promising results that demonstrate a significant improvement in the use of this new approach.
    }
    \vspace{\onelineskip}

    \noindent
    \emph{	
      \textbf{Keywords}: PSO; Particle Swarm Optimization; FJSP; Flexible Job Shop Problem; Population Intelligence; Dynamic components.
    }
  \end{otherlanguage*}
\end{resumo}
